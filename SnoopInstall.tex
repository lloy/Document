\documentclass[10pt]{article}
\usepackage[colorlinks,linkcolor=black]{hyperref}
\usepackage[infoshow]{tabularx}
\usepackage{times}
\usepackage{fontspec}
\usepackage{enumerate}
\usepackage{indentfirst}
\usepackage{graphicx}
\usepackage{listings}
\usepackage{zhfontcfg}
\usepackage{color}
% \usepackage[dvips]{color}
\newcommand{\itab}[1]{\hspace{1em}\rlap{#1}}
\newcommand{\tab}[1]{\hspace{.2\textwidth}\rlap{#1}}

\date{\today}

\lstset{
  language=Python,                % choose the language of the code
  basicstyle=\footnotesize,       % the size of the fonts that are used for the code
  numbers=left,                   % where to put the line-numbers
  numberstyle=\footnotesize,      % the size of the fonts that are used for the line-numbers
  stepnumber=1,                   % the step between two line-numbers. If it is 1 each line will be numbered
  numbersep=5pt,                  % how far the line-numbers are from the code
  backgroundcolor=\color{white},  % choose the background color. You must add \usepackage{color}
  showspaces=false,               % show spaces adding particular underscores
  showstringspaces=false,         % underline spaces within strings
  showtabs=false,                 % show tabs within strings adding particular underscores
  frame=single,                   % adds a frame around the code
  tabsize=2,              % sets default tabsize to 2 spaces
  captionpos=b,                   % sets the caption-position to bottom
  breaklines=true,        % sets automatic line breaking
  breakatwhitespace=false,    % sets if automatic breaks should only happen at whitespace
%  escapeinside={\%}{)}          % if you want to add a comment within your code
  escapeinside={} 
}

\setcounter{tocdepth}{3}

\setlength{\topmargin}{0cm}

%opening
\title{流量部署说明文档 }
\author{Hardy.zheng\thanks{首都在线 version 0.1}}

\begin{document}
\maketitle

\tableofcontents
\newpage

\section{土城抓包机器 部署说明}
\subsection{机器TC-Snooper说明(抓包机器)}
\begin{enumerate}
\item 系统
\begin{center}
\begin{tabular}{|p{10cm}|}
\hline
\hei{Distributor ID}:\tab {Ubuntu} \\
\hei{Description}:	\tab {Ubuntu 12.04.2 LTS}\\
\hei{Release}:	\tab {12.04}\\
\hei{Codename}:	\tab {precise}\\
\hline
\end{tabular}
\end{center}
	
\item PF\_RING版本\\
    \hei{version:} \tab {5.6.0}
\item pmacct版本 \\
    \hei{version:}\tab {1.5.0rc2}

\item 抓包统计 \\
Job模块下的文件: 
\begin{itemize}
    \item[*] job.py
    \item[*] remotion.py
    \item[*] settings.py
\end{itemize}
\end{enumerate}

\subsection{机器TC-IpCount(流量统计机器)}
\subsubsection{ 系统}
\begin{center}
\begin{tabular}{|p{10cm}|}
\hline
\hei{Distributor ID}:\tab {Ubuntu} \\
\hei{Description}:	\tab {Ubuntu 12.04.2 LTS}\\
\hei{Release}:	\tab {12.04}\\
\hei{Codename}:	\tab {precise}\\
\hline
\end{tabular}
\end{center}
	
\subsubsection{流量统计模块}	
\begin{enumerate}
\item Ipcount模块文件
\begin{itemize}
    \item[*] check\_subnet.py
    \item[*] ipcount.py
    \item[*] mongodb.py 
    \item[*] rebuild.py
    \item[*] settings.py
    \item[*] t1.py
\end{itemize}
\item Web 模块文件
\begin{itemize}
    \item[*] index.html
    \begin{itemize}
        \item[-] main.wsgi
        \item[-] none.png
    \end{itemize}
\end{itemize}
\end{enumerate}

\newpage

\section{机器TC-Snooper部署}
\subsection{基础安装}
\subsubsection{源处理}
\begin{center}
\begin{tabular}{|p{10cm}|}
\hline
\footnotesize {sudo apt-get remove :`{}dpkg -{}- get-selections | grep  i386 | awk '\{print \$1\}'`}
\\\hline
\end{tabular}
\end{center}
\itab{\hei{/etc/dpkg/dpkg.cfg.d/multiarch 里用\#屏蔽唯一一行}}
\subsubsection{下载包}
主要安装一些编译工具和下载一些安装包
\begin{enumerate}
    \item sudo apt-get install gcc g++ automake make autoconf python-pymongo python-mysqldb -y
    \item sudo apt-get install mysql-server mysql-client libmysqlclient-dev -y 数据库的用户名/秘密: root/cds-china
    \item mkdir Install;cd Install
    \item wget http://www.pmacct.net/pmacct-1.5.0rc2.tar.gz;tar xvf pmacct-1.5.0rc2.tar.gz
    \item wget http://sourceforge.net/projects/ntop/files/PF\_RING/Old/PF\_RING-5.6.0.tar.gz/download;tar xvf download
\end{enumerate}

\subsection{igb千兆网卡模块替换}
使用pf\_ring中的igb模块替换Ubuntu原有的igb模块,做法如下:
\begin{enumerate}
    \item cd PF\_RING-5.6.0
    \item cd drivers/PF\_RING\_aware/intel/igb/igb-4.1.2/src/;make;sudo make install
\end{enumerate}
\bigskip
\subsection{ixgbe 万兆网卡模块替换}
	使用pf\_ring中的ixgbe模块替换Ubuntu原有的igb模块,做法如下:
    \begin{enumerate}
		\item cd PF\_RING-5.6.0
		\item cd drivers/PF\_RING\_aware/intel/ixgbe/ixgbe-3.11.33/src/;make;sudo make install
    \end{enumerate}
\bigskip
\subsection{pf\_ring模块 安装}
\begin{enumerate}
	\item cd PF\_RING-5.6.0
	\item cd kernel;make;sudo make install
\end{enumerate}
查看/lib/modules/3.5.0-23-generic/kernel/net/pf\_ring/pf\_ring.ko 是否存在,存在则安装成功
\bigskip
\subsection{pf\_ring 用户空间lib库安装}
\begin{enumerate}
    \item cd userland/lib
    \item ./configure -{}-prefix=/usr;make;sudo make install
    \end{enumerate}
	查看下面文件存在则安装成功
    \begin{itemize}
        \item[*] /usr/lib/libpfring.a 
        \item[*] /usr/lib/libpfring.so
    \end{itemize}
\bigskip
\subsection{libpcap库替换安装}
\begin{enumerate}
    \item cd PF\_RING-5.6.0
    \item cd userland/libpcap-1.1.1-ring;./configure -{}-prefix=/usr;make;sudo make install
\end{enumerate}
	查看下面文件存在则安装成功:
    \begin{itemize}
        \item[*] libpcap.a 
        \item[*] libpcap.so 
		\item[*] libpcap.so.1 
		\item[*] libpcap.so.1.1.1 
    \end{itemize}
\bigskip
\subsection{pmacct安装}
\begin{enumerate}
    \item cd pmacct-1.5.0rc2
    \item  ./configure -{}-prefix=/usr/local -{}-enable-mysql -{}-with-mysql-libs=/usr/lib/x86\_64-linux-gnu/ -{}-prefix=/usr/local/pmacct/;make;sudo make install
    % \item echo "export PATH=$PATH:/usr/local/pmacct/bin:/usr/local/pmacct/sbin" >> ~{}/.bashrc
    % \item  source ~{}/.bashrc
	\item 将配置文件和启动脚步处理:
        \begin{itemize}
            \item[*] sudo mkdir /etc/pmacctd;cp Install/Job/etc/pmacctd/* /etc/pmacctd
            \item[*] sudo cp Install/Job/etc/init.d/pmacctd /etc/init.d/;sudo chmod +x /etc/init.d/pmacctd
        \end{itemize}
\end{enumerate}

\bigskip
\subsection{mysql 设置}
\begin{enumerate}
\item cd pmacct-1.5.0rc2/sql
\item  mysql -u root -p < pmacct-create-db\_v1.mysql
\item mysql -u root -p < pmacct-grant-db.mysql
\end{enumerate}

\bigskip
\subsection{Job 模块安装}
\hei{在抓包机器上部署}\\
将Job目录的文件复制到/usr/local/pmacct/bin/,如下:\\

\begin{center}
\begin{tabular}{|p{10cm}|}
\hline
sudo cp -fv Install/Job/* /usr/local/pmacct/bin/
\\\hline
\end{tabular}
\end{center}

\bigskip
\subsection{开机启动设置}
\begin{enumerate}
    \item igb, ixgbe, pf\_ring 模块开机自动加载:
		在/etc/modules文件里加入3行:
			igb
			ixgbe
			pf\_ring
        \item pmacctd 启动
		update-rc.d pmacctd defaults
		也可以通过执行行脚步 /etc/init.d/pmacctd start | stop
		
    \item Job开机启动
		在/etc/crontab 加入一行:
			*/1 *	* * *	root	python /usr/local/pmacct/bin/job.py
		在/etc/rc.local文件中加入一行:
			crontab /etc/crontab
        \end{enumerate}

\bigskip
\subsection{重新启动查看模块是否启动}
\begin{enumerate}
    \item igb: lsmod | grep igb
    \item ixgbe: lsmod|grep ixgbe
    \item pf\_ring: lsmod|grep pf\_ring
    \item pmacctd: ps aux|grep pmacctd or ls /var/run/pmacctd*
    \item sudo crontab -l -u root
\end{enumerate}
\newpage


\section{TC-IpCount机器部署}
\subsection{基础}
\subsubsection{源处理}
    sudo apt-get remove `dpkg -{}-get-selections | grep i386 | awk '\{print \$1\}'` \\
/etc/dpkg/dpkg.cfg.d/multiarch 里用\#屏蔽唯一一行 \\
\subsubsection{下载工具包}
\begin{enumerate}
	\item sudo apt-get install gcc g++ automake make autoconf -y
	\item sudo apt-get install chkconfig -y	
	\item sudo apt-get install python-pymongo mongodb-server python-mysqldb -y
	\item sudo apt-get install python-rrdtool -y 
	\item sudo apt-get install python-pip -y
	\item sudo pip install ipaddress 
\end{enumerate}

\subsubsection{设置环境变量}
export LC\_ALL="C" 
\bigskip

\subsection{IpCount安装}
sudo mkdir /usr/local/cloudcount \\
将IpCount目录的文件复制到/usr/local/pmacct/bin/,如下: \\

\begin{center}
\begin{tabular}{|p{10cm}|}
\hline
    sudo cp -fv Install/IpCount/* /usr/local/cloudcount/
\\\hline
\end{tabular}
\end{center}

\bigskip
\subsection{Web安装}
\begin{enumerate}

    \item sudo mkdir /var/www
    \item sudo cp -rv Install/Web/index.html /var/www/
    \item sudo cp -rfv Install/Web/wsgi /var/www/
\end{enumerate}


\bigskip
\subsection{开机启动}
\subsubsection{Ipcount自动启动} 
		在/etc/crontab 加入一行: \\
			*/5 *	* * *	root	python /usr/local/cloudcount/ipcount.py\\
		在/etc/rc.local文件中加入一行: \\
			crontab /etc/crontab
\subsubsection{mongodb启动}

\begin{center}
\begin{tabular}{|p{10cm}|}
\hline
		sudo update-rc.d mongodb defaults 
\\\hline
\end{tabular}
\end{center}
通过 chkconfig --list | grep mongo 查看mongodb的启动方式

		
\bigskip

\bibliography{}
\clearpage
\end{document}

